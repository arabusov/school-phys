\documentclass[a5paper,11pt]{article}
%\semiisopage
\usepackage{amssymb,amsmath,amsthm}
%\pagestyle{companion}
\usepackage[T2A]{fontenc}
\usepackage[utf8]{inputenc}
\usepackage[russian]{babel}
\usepackage{indentfirst}
\usepackage{graphicx,caption}
\graphicspath{{images/}}
\newtheorem{definition}{Определение}
\newtheorem{axiom}{Постулат}
\newtheorem{theorem}{Теорема}
\newtheorem{example}{Пример}
\newtheorem{task}{Задача}
\newcommand{\solution}{\proof[Решение]}
\title{Кинематика}
\author{Андрей Рабусов (\texttt{a.rabusov@tum.de})}
\date{25 апреля 2022г.}
\begin{document}
\maketitle
\section{Основные понятия}
Будем исследовать движение материальной точки безотносительно
причин возникновения движения. Всякая материальная точка располагается в
трёхмерном пространстве. Введём понятие координаты материальной точки
как вектор в таком трёхмерном пространстве $\vec{r}$. Такой вектор ещё называют
радиус-вектором. В таком пространстве можно ввести ортонормированный базис
$\{\vec{i},$ $\vec{j},$ $\vec{k}\}$,
поэтому координату материальной точки можно разложить по этому базису
$\vec{r} = x\vec{i}+y\vec{j}+z\vec{k}.$ В классической механике 
можно ввести время как действительный параметр $t\in\mathbb{R},$
при этом компоненты радиус-вектора материальной точки становятся
функциями от времени, то есть $x = x(t),~y=y(t),~z=z(t).$
Формально это можно записать как $\vec{r} = \vec{r} (t),$ то есть
ввести зависимость радиус-вектора от времени.
\begin{definition}[Средняя скорость]
$\amalg$ даны моменты времени $t_1,$ $t_2,$ $t_1 < t_2.$
Тогда средняя скорость определяется как
$$\vec{v}_{av} = \frac{\vec{r}(t_2)-\vec{r}(t_1)}{t_2-t_1}.$$
\label{def:mean_speed}
\end{definition}
В определинии~\ref{def:mean_speed} присутствуют такие выражения, как
$\vec{r}(t_2)-\vec{r}(t_1)$ и $t_2 - t_1.$ Обозначим изменение некой
величины $a$ за время от $t_1$ до $t_2$ за $\Delta a.$ Тогда определение
средней скорости перепишется как $$\vec{v}_{av} = \frac {\Delta \vec{r}}{\Delta t}.$$
\par
Некоторые функции $f(t)$ можно представить в виде бесконечного ряда
$f(t) = c_0(t_0)+c_1(t_0) (t-t_0) + c_2(t_0) (t-t_0)^2 + \ldots.$ В механике обычно
пользуются только такими функциями, что даёт нам право рассматривать $\vec{r}(t)$
в виде ряда, где за $t_0$ выбирается некоторый конкретный момент времени.
Давайте рассмотрим теперь определение средней скорости в предположении,
что
$\vec{r}(t) = \vec{c}_0(t_0) +\vec{c}_1(t_0) (t-t_0) + \vec{c}_2(t_0) (t-t_0)^2+\ldots.$
Подставим это выражение в определение средней скорости, заменяя $t$ на $t_1$ или $t_2,$
получим
\begin{multline*}
\vec{v}_{av} =
 \frac{\vec{c}_0(t_0) +\vec{c}_1(t_0) (t_2-t_0) + \vec{c}_2(t_0) (t_2-t_0)^2+\ldots}{t_2-t_1}-\\
 \frac{\vec{c}_0(t_0) +\vec{c}_1(t_0) (t_1-t_0) + \vec{c}_2(t_0) (t_1-t_0)^2+\ldots}{t_2-t_1}=\\
 \frac{\vec{c}_1(t_0) (t_2-t_1) + \vec{c}_2(t_0) (t_2-t_1)^2+\ldots}{t_2-t_1}.
\end{multline*}
Окончательно имеем
\begin{equation}
\vec{v}_{av} = \vec{c}_1(t_0) + \vec{c}_2(t_0)(t_2-t_1) +  \ldots.
\label{eq:v_av}
\end{equation}
Теперь сделаем предельный переход, а именно получим выражение для средней
скорости при $t_0 = t,$ при этом $t_1 \to t$ и $t_2 \to t.$ Так как
(\ref{eq:v_av}) не содержит особенностей при $t_1 = t_2,$ то можно
положить $t_1 = t_2 = t,$ откуда получим
$$\vec{v} = c_1 (t).$$
Таким образом записанную среднюю скорость называют мгновенной скоростью, то есть
такой скоростью, что $\Delta t \to 0.$ 
Кратко проделанная процедура записывается
как $\lim\limits_{\Delta t \to 0}.$ 
Теперь мы всё имеем для того, чтобы дать определение
мгновенной скорости.
\begin{definition}[Мгновенная скорость]
$$\vec{v}(t) = \lim\limits_{\Delta t \to 0} \frac {\Delta \vec{r}}{\Delta t}.$$
\end{definition}
Обратите внимание на то, что мгновенная скорость
сама является функцией от $t,$ также мгновенная скорость --- это вектор, поэтому
её так же можно разложить на компоненты в пространстве с базисными векторами $\{\vec{i},\vec{j},\vec{k}\}.$
Мгновенная скорость достаточно часто встречается в задачах, поэтому разумно
называть её также скоростью.
Проделанную выше процедуру можно повторить для скорости $\vec{v}.$
\begin{definition}[Ускорение]
$$\vec{a}(t) = \lim\limits_{\Delta t \to 0} \frac {\Delta \vec{v}}{\Delta t}.$$
\end{definition}
Величина $\vec{a}(t)$ --- векторная функция от времени, её также можно покомпонентно разложить
в пространстве с базисом $\{\vec{i}, \vec{j}, \vec{k}\}.$
\par
В изложенном выше подходе мы делали разложение в ряд, а затем клали  $t_0 = t.$ Однако в дальнейшем
мы бы хотели оставить $t_0$ неизменным и заданным, например, условиями задачи и исходя
из известной вектор-функции $\vec{r}(t)$ получить выражение для скорости и ускорения в явном
виде. Давайте сначала рассмотрим случай, когда $\vec{r} = \vec{c}_n (t_0) (t-t_0)^n.$
Тогда $\Delta \vec{r} = \vec{c}_n (t+\Delta t -t_0)^n-(t-t_0)^n.$ Для того, чтобы
раскрыть скобки, нам понадобится равенство 
\begin{equation}
(q-1)\cdot \sum\limits_{k=0}^{n-1}q^k = q^n-1,
\label{eq:sum_geom}
\end{equation}
где $q \in \mathbb{R}_+.$ Имеем
$$\Delta \vec{r} = \vec{c}_n (t-t_0)^n\left(\frac{t+\Delta t-t_0}{t-t_0} - 1\right)^n,$$
откуда по (\ref{eq:sum_geom}) получим
$$\Delta \vec{r} = \vec{c}_n (t-t_0)^n\left(\frac{t+\Delta t-t_0}{t-t_0} - 1\right)\cdot 
				\sum\limits_{k=0}^{n-1}\left(\frac{t+\Delta t-t-0}{t-t_0}\right)^k.$$
Умножим сумму на $(t-t_0)^{n-1},$ дробь умножим на $(t-t_0),$ получим
$$\Delta \vec{r} = \vec{c}_n \left(t+\Delta t-t_0-t+t_0\right)\cdot 
				\sum\limits_{k=0}^{n-1}\left(t+\Delta t-t_0\right)^k\left(t-t_0\right)^{n-1-k}.$$
Подставим полученное выражение в выражение для скорости
$$\frac{\Delta \vec{r}}{\Delta t} = \vec{c}_n  
				\sum\limits_{k=0}^{n-1}\left(t+\Delta t-t_0\right)^k\left(t-t_0\right)^{n-1-k}$$
и положим $\Delta t = 0$ (предельный переход):
$$\lim\limits_{\Delta t \to 0}\frac{\Delta \vec{r}}{\Delta t} = \vec{c}_n  
				\sum\limits_{k=0}^{n-1}\left(t-t_0\right)^k\left(t-t_0\right)^{n-1-k}.$$
Окончательно имеем
\begin{equation}
\vec{v} = \vec{c}_n n(t-t_0)^{n-1}
\label{eq:diff_pow}
\end{equation}
Теперь представим $\vec{r}$ в виде многочлена: $\vec{r} = \sum\limits_{k=0}^{n}\vec{c}_k(t-t_0)^k.$
$\Delta \vec{r} =$ $
\Delta\sum\limits_{k=0}^{n}\vec{c}_k(t-t_0)^k$
$= \sum\limits_{k=0}^{n}\Delta\left(\vec{c}_k(t-t_0)^k\right).$
По определению скорости
$$\lim\limits_{\Delta t \to 0}\frac{\Delta \vec{r}}{\Delta t} 
= \sum\limits_{k=0}^{n}\lim\limits_{\Delta t \to 0}\frac{\Delta\vec{c}_k(t-t_0)^k}{\Delta t}.$$
Под знаком суммы получилось точно такое же выражение, какое мы рассматривали в предыдущем случае,
поэтому применим сразу выражение (\ref{eq:diff_pow}), получим
$\vec{v} = \sum\limits_{k=0}{n}\vec{c}_k k(t-t_0)^{k-1}.$
Заметим, что при $k=0$ выражение под
знаком суммы зануляется. Введём новую переменную суммирования $l=k-1,$ окончательно получим
\begin{equation}
\vec{v} = \sum\limits_{l=0}^{n-1}\vec{c}_{l+1}(l+1)(t-t_0)^{l}.
\label{eq:teilor_speed}
\end{equation}
Для ускорения можно проделать аналогичную последовательность действий. Предлагаю
читателю самостоятельно убедиться в справедливости выражения
\begin{equation}
\vec{a} = \sum\limits_{l=0}^{n-2}\vec{c}_{l+2}(l+2)(l+1)(t-t_0)^{l}.
\label{eq:teilor_accel}
\end{equation}
В некоторых случаях удаётся обобщить такой результат при $n \to +\infty.$ Подробно вопросы
существования таких рядов рассматриваются в курсе теории функций комплексного переменного.
\par
Подведём некоторый итог. В каждый момен времени $t$ материальная точка характеризуется
своим радиус-вектором $\vec{r},$ своей скоростью $\vec{v}$ и своим ускорением $\vec{a}.$
Зная значение радиус-вектора в каждый момент времени, можно получить значение скорости
и ускорения в каждый момент времени. Зная скорость в каждый момент времени, можно
найти ускорение в каждый момент времени, а зная также начальный радиус-вектор, можно найти
его в каждый момент времени. Знание ускорения, начальной скорости и начального радиус-вектора
позволяет найти скорость и радиус-вектор в любой момент времени. Это основные задачи
кинематики, на которых мы подробно остановимся. Однако перед этим дадим некоторое количество
определений, которые могут понадобиться в задачах.
\begin{definition}[Траектория]
Множество точек пространства, в которых находилась материальная точка.
Знание $\vec{r}(t) \forall t$ автоматически влечёт за собой знание
траектории.
\end{definition}
\begin{definition}[Длина пути]
Длина траектории
\end{definition}
\begin{definition}[Перемещение]
$\amalg$ $t_1 < t_2.$ Тогда $\vec{s} = \Delta {\vec{r}}.$
\end{definition}
\subsection{Равномерное движение}
Вооружившись знаниями из предыдущего раздела, мы можем теперь решать основную задачу кинематики
для некоторых частных случаев.
\begin{definition}[Равномерное движение] $\vec{v}(t) = \overrightarrow{const}$\end{definition}
Пусть в начальный момент времени задан радиус-вектор: $\vec{r}(t_0) = \vec{r}_0$.
Представим радиус-вектор в виде многочлена:
$\vec{r}(t) = \vec{c}_0(t_0) +\vec{c}_1(t_0) (t-t_0) + \vec{c}_2(t_0) (t-t_0)^2+\ldots.$
В момент времени $t_0$ $\vec{r}(t_0) = \vec{c}_0(t_0),$ с другой стороны по условию имеем
$\vec{r}(t_0) = \vec{c}_0 (t_0),$ поэтому $\vec{c}_0 = \vec{r}_0.$
Воспользуемся выражением (\ref{eq:teilor_speed}). В любой момент времени
скорость постоянна и равна $\vec{v}.$ С другой стороны, скорость выражается как
$\vec{v} = \sum\limits_{l=0}^{n-1}\vec{c}_{l+1}(l+1)(t-t_0)^{l}.$ Ясно, что
при $l>1$ $\vec{c}_l = \vec{0}.$ $\vec{c}_1=\vec{v},$ откуда получим
\begin{equation}
\vec{r} (t) = \vec{r}_0 + \vec{v}(t-t_0).
\label{eq:const_speed}
\end{equation}
Часто начальный момент времени выбирают при $t_0 = 0,$ тогда
формула (\ref{eq:const_speed}) упрощается.
\subsection{Равноускоренное движение}
\begin{definition}[Равноускоренное движение] $\vec{a}(t) = \overrightarrow{const}$\end{definition}
Будем действовать аналогично предыдущему пункту. Кроме заданного постоянного ускорения, нам также
потребуется знание в момент времени $t_0$ величин скорости $\vec{v}_0$ и радиус-вектора $\vec{r}_0.$
Представим $\vec{r}$ в виде многочлена, $\vec{c}_0 = \vec{r}_0.$ Используя для скорости
выражение (\ref{eq:teilor_speed}) получим, что $\vec{c}_1 = \vec{v}_0.$ Используя
для ускорения выражение (\ref{eq:teilor_accel}) получим, что
$\vec{c}_2 = \vec{a}/2,$ при этом при $l>2$ $\vec{c}_l = 0.$ Окончательно имеем
\begin{subequations}
\begin{align}
\vec{v}& = \vec{v}_0 + \vec{a} (t-t_0),\label{eq:const_accel_speed}\\
\vec{r}& = \vec{r}_0 + \vec{v}_0 (t-t_0) + \vec {a} \frac{(t-t_0)^2}{2}\label{eq:const_accel_coord}.
\end{align}
\end{subequations}
Также как и в предыдущем пункте мы можем положить за начальный момент времени
$t_0 = 0.$

\section{Относительность движения}
В процессе развития естествознания происходил постепенный переход от концепций, где
существовал некий абсолютный центр Вселенной, относительно которого происходит движение.
Современная концепция такова: не существует абсолютного центра, абсолютной системы отсчёта:
все точки пространства эквивалентны между собой, также эквивалентны все направления в пространстве.
Кроме того, утверждается, что все моменты времени эквивалентны. Эти принципы называются
однородностью пространства, изотропией пространства и однородностью времени. Пока экспериментально
не удалось обнаружить каких-то значительных отклонений от данных принципов. Тем не менее, вопрос
о верности данных утверждений открыт. Мы примем данные принципы за постулаты.

Очень тесно связанным с предыдущими принципами является относительность движения в различных
инерциальных системах отсчёта. Формулируется он следующим образом: все инерциальные системы отсчёта
равноправны между собой.
\begin{definition}[Инерциальная система отсчёта]
Система отсчёта (совокупность системы координат и матераильной точки, с которой эта система связана),
которая движется равномерно прямолинейно.
\end{definition}
Данная эквивалентность сводится к тому, что, во-первых, мы можем решать задачу в любой, наиболее удобной,
системе отсчёта, лишь бы она была инерциальной. Во-вторых, не существует таких понятий, как абсолютная
координата, скорость, траектория и подобные. Все эти понятия требуют задания инерциальной системы отсчёта.
Так как оказывается, что некоторые физические понятия меняются при переходе из одной системы отсчёта в другую,
следует вывести правила перехода между инерциальными системами отсчёта. Такие правила в классической механике называются
преобразованиями Галилея.
\subsection{Преобразования Галилея}
Рассмотрим две инерциальные системы отсчёта, пусть одна движется относительно
другой со скоростью $\vec{u},$ оси сонаправлены. Обозначим покоющуюся систему отсчёта
за $\mathbb{S},$ движущуюся за $\mathbb{S}^\prime.$ Пусть дана материальная точка с некой
координатой $\vec{r}^\prime$  в системе отсчёта $\mathbb{S}^\prime.$
Постулируется одинаковое протекание времени в различных системах отсчёта, то есть
$t = t^\prime$. Несложные геометрические построения приводят к тому, что
\begin{equation}
\vec{r} = \vec{u} t^\prime + \vec{r}^\prime
\label{eq:Galilei-main}
\end{equation}
Это и есть преобраования Галилея. Получим некоторые следствия из них.\par
\textbf{Сложение скоростей.} $\vec{v} = \vec{u} + \vec{v}^\prime$, получается
предельным переходом 
$\lim\limits_{\Delta t \to 0}\frac{\Delta}{\Delta t}$
из выражения (\ref{eq:Galilei-main}).\par
\textbf{Преобразования ускорения.} $\vec{a} = \vec{a}^\prime$, получается
двойным предельным переходом 
$\lim\limits_{\Delta t \to 0}\frac{\Delta}{\Delta t}$
из выражения (\ref{eq:Galilei-main}).\par
Также важным следствием преобразований Галилея является независимость расстояний
при переходе из одной инерциальной системы отсчёта в другую.
\section{Решение кинематических задач }
Необходимым условием кинематической задачи является отсутствие динамических терминов, таких, как
<<масса>>, <<сила>>, <<импульс>>. 
Часто динамические задачи сводятся к кинематическим, поэтому
умение решать задачи на кинематику является одним из самых важных в механике. 
\subsection{Задачи}
\begin{task}
Материальная точка в начальный момент времени обладает скоростью $v$ и углом между
направлением скорости и горизонтом $\varphi,$ движется в поле тяжести $g$. Найдите
дальность полёта, максимальную высоту полёта.
\label{task:k-95-1}
\end{task}
\begin{task}
Тело падает без начальной скорости с высоты $H.$ Найти среднюю скорость падения на верхней
половине пути.
\label{task:k-96-1}
\end{task}
\begin{task}
Автомобиль начал двигаться равноускоренно и, пройдя некоторое расстояние, достиг скорости $v.$
Какова была скорость машины на половине этого пути?
\label{task:k-97-1-1}
\end{task}
\begin{task}
С вершиныы холма бросили камень под углом к горизонту со скоростью $v = 10\text{м/с}.$ В момент падения
угол  между направлением скорости камня и горизонтом составляет $\beta = \pi/3,$ а разность высот точек
бросания и падения равна $\Delta h = 5 \text{м}.$ Найдите угол между направлением начальной скорости
камня и горизонтом.
\label{task:k-97-4-1}
\end{task}
\begin{task}
Тело начинает двигаться вдоль прямой без начальной скорости с постоянным ускорением. Через время
$t=30\text{мин}$ ускорение тела меняется по направлению на противоположное, оставаясь таким же по величине.
Через какое время от начала движения тело вернётся в сходную точку?
\label{task:k-97-1-2}
\end{task}
\begin{task}
Тело брошено горизонтально со скоростью $v.$ На какой высоте брошено тело, если оно
упало на землю под углом $\beta = \pi/3$ к углу к горизонту?
\label{task:k-98-1-1}
\end{task}
\begin{task}
Кусочек мела положили на движущуюся с постоянной скоростью ленту транспонтёра. Сначала мел скользил по ленте.
Когда проскальывание прекратилось, он прочертил на ней линию длины L. На какое расстояние переместился мел к этому
моменту?
\label{task:k-06-1}
\end{task}
\begin{task}
Гора имеет форму трапеции: её левый склон назодится под углом $\alpha$ к горизонту, а плоская горизонтальнся вершина
имеет протяжённость $L.$ Автомобиль ехал по левому склону, перелетел через вершину и мягко, без удара о
землю, приземлился на правый склон. Какой была скорость автомобиля на левом склоне? Под каким углом к горизонту
располагается правый склон?
\label{task:k-06-2}
\end{task}
\begin{task}
Автомобиль, двигающийся со скоростью $u$ перпендикулярно стене, издаёт звуковой сигнал
длительностью $\tau.$ Найдите длительность сигнала, услышанного автомобилистом
после отражения звука от стены. Скорость звука $c.$
\label{task:k-07-1-8}
\end{task}
\begin{task}
Мяч, брошенный со скоростью $u$ под углом $\alpha$ к горизонту, ударяется о вертикалную стену.
Расстояние до стены $l$, удар упругий. На какой высоте мяч ударится о стену и на каком расстоянии от
стены он упадёт на пол? Сопротивлением воздуха пренебречь.
\label{task:k-07-4-8}
\end{task}
\begin{task}
По шоссе в обе стороны движутся одинаковые потокм машин со скоростью $v=60\text{км/ч}.$
Велосипедист встречает 3 машины в минуту, а обгоняют его 2 машины в минуту. С какой скоростью
едет велосипедист?
\label{task:k-08-1-9}
\end{task}
\begin{task}
Небольшое тело съезжает без трения по горке длины $L$ с углом наклона $\alpha.$ На высоте
$h$ над землёй горка обрывается. Найти точку падения тела.
\label{task:k-08-2-9}
\end{task}
\begin{task}
Первую половину пути автомобиль разгоняется из состояния покоя с постоянным ускорением, а вторую проезжает
с постоянной скоростью $v.$ Чему равна средняя скорость?
\label{task:k-08-1-10}
\end{task}
\begin{task}
Лодочник, переправляясь через реку шириной $H$ из пункта $A,$ всё время направляет
лодку перпендикулярно берегу. Определить скорость лодки относительно воды $v,$ если скорость
течения реки $u,$ а лодку снесло ниже пункта $B$ на расстояние $L.$
\label{task:k-09-1-9}
\end{task}
\begin{task}
Камень, брошенный с начальной скоростью $v$, упал на землю через время $t$. Найдите начальный угол
и дальность полёта камня.
\label{task:k-09-2-9}
\end{task}
\begin{task}
Лодочник,перенаправляясь через реку шириной $H$ из пункта $A$, всё время направляет лодку
под углом $\alpha$ к течению. Определите скорость лодки относительно воды $v$, если скорость
течения реки $u,$ а лодку снесло ниже пункта $B$ на расстояние $L$.
\label{task:k-09-1-10}
\end{task}
\begin{task}
На длинной спице, установленной под углом $\alpha$ к горизонту, в поле тяжести могут скользить
без трения две бусинки. Начальное расстояние между бусинками равно $L$. Нижней бусинке придают
скорость $v$ навстречу верхней, а верхней придают скорость $v/2$ в направление от нижней. В какой точке
относительно нижней бусинки произойдёт столкновение.
\label{task:k-11-1}
\end{task}
\end{document}
