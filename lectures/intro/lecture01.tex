\documentclass[t,aspectratio=169]{beamer}
\usepackage[T2A]{fontenc}
\usepackage[utf8]{inputenc}
\usepackage[russian,english]{babel}
\usepackage{graphicx}
\usepackage{commath}
\usepackage{tikz}
\usepackage{tikz-feynman}
\usepackage{tikz-3dplot}
\usetikzlibrary{arrows,automata,backgrounds,calendar,chains,matrix,mindmap,
    patterns,petri,shadows,shapes.geometric,shapes.misc,spy,trees,calc}
\usepackage{verbatim}
\usepackage{listings}

%%% Some definitions for math
\usepackage{amsmath}
\usepackage{bbm}

\setbeamertemplate{navigation symbols}{}
\setbeamertemplate{footline}[frame number]

\usepackage{comment}
\usepackage{url}
\usepackage{mathtools}
\usefonttheme[onlymath]{serif}
\graphicspath{{./images/}}
\title{Физика}
\subtitle {Введение}
\institute[TUM E18]{Technical University of Munich\\
Department of Physics, E18}
\author[Андрей Рабусов~\url{a.rabusov@tum.de}]{\texorpdfstring{Андрей Рабусов \newline\url{a.rabusov@tum.de}}{Андрей Рабусов}}

\date{13 Апреля 2022}


\newcommand{\backupbegin}{
\newcounter{finalframe}
\setcounter{finalframe}{\value{framenumber}}
}
\newcommand{\backupend}{
\setcounter{framenumber}{\value{finalframe}}
}
\begin{document}
\pdfinfo{
/Author (Andrei Rabusov)
/Title  (physics)
/Subject (intro)
}

\begin{frame}
\maketitle
\end{frame}
\begin{frame}{Содержание}
\tableofcontents[sectionstyle=show,subsectionstyle=show]
\end{frame}
\section{Введение}
\begin{frame}
    \frametitle{Сферы высказываний}
    \begin{columns}
        \begin{column}{0.5\textwidth}
            В аксиоматике социологии выделяют различные автономные сферы
            человеческого действия по принципам:
            \begin{itemize}[<+->]
                \item Свой язык
                \item Своя логика
                \item Своя рациональность
                \item Свой способ производства высказываний и действий
            \end{itemize}
        \end{column}
        \begin{column}{0.5\textwidth}
            Замкнутые системы согласно этой аксиоматике:
            \begin{itemize}[<+->]
                \item Философия
                \item Религия
                \item Искусство
                \item Политика
                \item Наука
                \item Право
                \item Хозяйствование
                \item Быт
            \end{itemize}
        \end{column}
    \end{columns}
\end{frame}

\begin{frame}
    \frametitle{Замкнутые системы}
    \only<1->{
        Философия: познание наиболее базовых принципов изучения мира, человека,
        и отношений между людьми
    }
    \only<2->{
        \par\vspace{0.3cm}
        Религия (обычно): отношение к непознаваемому
    }
    \par\vspace{0.3cm}
    \only<3->{
        Искусство: образное осмысление действительности
    }
    \only<4->{
        \par\vspace{0.3cm}
        Политика: совершение активных действий для изменения или консервации
        текущего положения дел
    }
    \only<5->{
        \par\vspace{0.3cm}
        Наука: познание мира
    }
\end{frame}


\begin{frame}
    \frametitle{Наука}
    \only<1->{
        \par\vspace{0.3cm}
        Математика: наука об отношениях между формальными объектами
    }
    \only<2->{
        \par\vspace{0.3cm}
        Лингвистика: наука о языках
    }
    \only<3->{
        \par\vspace{0.3cm}
        Социология: наука об отношениях между группами людей
    }
    \only<4->{
        \par\vspace{0.3cm}
        Психология: наука о психической деятельности людей и
        групп людей
    }
    \only<5->{
        \par\vspace{0.3cm}
        Химия: наука о веществе на молекулярном уровне
    }
    \only<6->{
        \par\vspace{0.3cm}
        Биология: наука о живой материи
    }
    \only<7->{
        \par\vspace{0.3cm}
        Музыкознание: наука о музыке
    }
\end{frame}

\begin{frame}
    \frametitle{Физика как наука о фундаментальных законах материи}
    Историческое развитие физики
    \begin{itemize}[<+->]
        \item Аристотель (IV век до н.э.): формулирует всеобщие законы
            движения, космологии, химии, биологии, и т.д.
        \item Галилей (XVII век): законы движения и экспериментальный
            метод
        \item Ньютон (XVII--XVIII): механика, математический анализ
        \item Кельвин, Больцман (XIX век): термодинамика и
            статистическая физика
        \item Фарадей и Максвелл (XIX век): электричество и магнетизм
        \item Эйнштейн (XX век): общая теория относительности
        \item Планк, Борн (XX век): квантовая механика
        \item Салам, Вайнберг (XX век): электрослабая теория
    \end{itemize}
\end{frame}
\begin{frame}
    \frametitle{Физика как наука о фундаментальных законах материи}
    Современные разделы физики:
    \begin{itemize}[<+->]
        \item Физика элементарных частиц
        \item Астрофизика
        \item Космология
        \item Атомная физика
        \item Квантовая оптика
        \item Физика плазмы
        \item Химическая физика
        \item Биофизика
    \end{itemize}
\end{frame}

\begin{frame}
    \frametitle{Карьера в физике}
    \only<1->{
        \par\vspace{0.3cm}
        Школа
    }
    \only<2->{
        \par\vspace{0.3cm}
        Бакалавриат (1--2 курсы)
    }
    \only<3->{
        \par\vspace{0.3cm}
        Выбор кафедры, бакалавриат (3--4 курсы)
    }
    \only<4->{
        \par\vspace{0.3cm}
        Магистратура (5--6 курсы)
    }
    \only<5->{
        \par\vspace{0.3cm}
        Аспирантура (от трёх лет до бесконечности)
    }
    \only<6->{
        \par\vspace{0.3cm}
        Post-doctoral research (рабочие лошидки науки)
    }
    \only<7->{
        \par\vspace{0.3cm}
        Профессура (боги науки)
    }
\end{frame}

\begin{frame}
    \begin{figure}
        \begin{centering}
            \includegraphics[width=0.6\textwidth]{sci}
        \end{centering}
    \end{figure}
\end{frame}

\input{tex/subatom}
\input{tex/mech}
\section{Заключение}
\begin{frame}
    \frametitle{Послесловие}
    \only<1->{
        Это был всего лишь обзорный взгляд на физику с точки зрения
        философствующего аспиранта.
    }
    \only<2->{
        \par\vspace{0.3cm}
        Настоящий вводный курс по механике должен состоять из пары лекций,
        пяти семинаров, и письменным и устным экзаменами в конце.
    }
    \only<3->{
        \par\vspace{0.3cm}
        Тем, кому интересно заниматься дальше: обратитесь ко мне, мы
        попробуем договориться с <<Абракадаброй>> и ТУМом.
    }
    \only<4->{
        \par\vspace{0.3cm}
        Заодно можно будет посмотреть на всякие интересные штуки в Гархинге
        или в Deutsches Museum.
    }
    \only<5->{
        \par\vspace{0.3cm}
        И хотелось бы мне вам рассказать про низкоуровневое
        программирование: как шаг за шагом конструируются современные
        компьютерные программы, начиная с уровня микрочипов.
    }
\end{frame}

\begin{frame}
    \frametitle{Спасибо за внимание!}
    \begin{figure}
        \begin{centering}
            \includegraphics[width=0.7\textwidth]{summary}
        \end{centering}
    \end{figure}
\end{frame}

\begin{frame}
    \frametitle{Ссылки на источники}
    Публичная лекция В. Вахштайна <<Наука и искусство: самореференция и
    интервенция>>
    \par\vspace{0.3cm}
    Открытая группа CERN на facebook
    \par\vspace{0.3cm}
    Открытая группа Belle II на facebook
    \par\vspace{0.3cm}
    Материалы с \url{https://belle2.org}
    \par\vspace{0.3cm}
    Личные фотографии
\end{frame}

\appendix
\backupbegin
\begin{frame}
\frametitle{Приложение}
\begin{figure}
    \LARGE Backup
\end{figure}
\end{frame}
%\input{tex/appendix}
\backupend
\end{document}
