\documentclass[t,aspectratio=169]{beamer}
\usepackage[T2A]{fontenc}
\usepackage[utf8]{inputenc}
\usepackage[russian,english]{babel}
\usepackage{graphicx}
\usepackage{commath}
\usepackage{tikz}
\usepackage{tikz-feynman}
\usepackage{tikz-3dplot}
\usetikzlibrary{arrows,automata,backgrounds,calendar,chains,matrix,mindmap,
    patterns,petri,shadows,shapes.geometric,shapes.misc,spy,trees,calc}
\usepackage{verbatim}
\usepackage{listings}

%%% Some definitions for math
\usepackage{amsmath}
\usepackage{bbm}

\setbeamertemplate{navigation symbols}{}
\setbeamertemplate{footline}[frame number]

\usepackage{comment}
\usepackage{url}
\usepackage{mathtools}
\usefonttheme[onlymath]{serif}
\graphicspath{{./images/}}
\title{Физика}
\subtitle {Симметрии}
\institute[TUM E18]{Technical University of Munich\\
Department of Physics, E18}
\author[Андрей Рабусов~\url{a.rabusov@tum.de}]{\texorpdfstring{Андрей Рабусов \newline\url{a.rabusov@tum.de}}{Андрей Рабусов}}

\date{18 Мая 2022}


\newcommand{\backupbegin}{
\newcounter{finalframe}
\setcounter{finalframe}{\value{framenumber}}
}
\newcommand{\backupend}{
\setcounter{framenumber}{\value{finalframe}}
}
\begin{document}
\pdfinfo{
/Author (Andrei Rabusov)
/Title  (physics)
/Subject (intro)
}

\begin{frame}
\maketitle
\end{frame}
\section{Симметрии}
\begin{frame}{Физика частиц: содержание}
    \tableofcontents[currentsection, subsectionstyle=show/show/hide]
\end{frame}
\subsection{Дискретные симметрии}
\begin{frame}
    \frametitle{Симметрия треугольника}
    \begin{columns}
        \begin{column}{0.3\textwidth}
            \begin{figure}[H]
                \begin{centering}
                    \begin{tikzpicture}
    \coordinate (x) at (0.87, -0.5);
    \coordinate (y) at (0.00, 1.00);
    \coordinate (z) at (-0.87, -0.5);
    \draw[thin] (x) -- (y);
    \draw[thin] (y) -- (z);
    \draw[thin] (z) -- (x);
    \draw (x) node[anchor=north west]{$1$};
    \draw (z) node[anchor=north east]{$3$};
    \draw (y) node[anchor=south]{$2$};
\end{tikzpicture}

                \end{centering}
            \end{figure}
        \end{column}
        \begin{column}{0.3\textwidth}
        \end{column}
        \begin{column}{0.3\textwidth}
        \end{column}
    \end{columns}
\end{frame}

\subsection{Непрерывные симметрии}
\begin{frame}
    \frametitle{Стандартная модель ФЭЧ}
    \begin{figure}
        \begin{centering}
            \includegraphics[width=0.7\textwidth]{sm}
        \end{centering}
    \end{figure}
\end{frame}

\end{document}
