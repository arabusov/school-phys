\documentclass[10,a5paper]{article}
\usepackage[T2A]{fontenc}
\usepackage[utf8]{inputenc}
%koi8-r]{inputenc}
\usepackage[english,russian]{babel}
\usepackage{amsmath}
\usepackage{amsfonts}
\usepackage{amssymb}
\usepackage[top=0.5cm,bottom=0.5cm]{geometry}

% \usepackage{color}
\usepackage[usenames,dvipsnames,svgnames,table]{xcolor}
\usepackage[colorlinks=true, linkcolor=Maroon]{hyperref}
\pagestyle{empty}


\newcommand*{\mc}{}
\def\mc#1#{\mathcoloraux{#1}}
\newcommand*{\mathcoloraux}[3]{%
  \protect\leavevmode
  \begingroup
    \color#1{#2}#3%
  \endgroup
}

\usepackage{wrapfig}


\usepackage{graphicx}
\usepackage{physics}
\usepackage{tikz}
\usepackage[outline]{contour} % glow around text
\usetikzlibrary{patterns,snakes}
\usetikzlibrary{arrows.meta} % for arrow size
\contourlength{0.4pt}

\colorlet{xcol}{blue!70!black}
\colorlet{darkblue}{blue!40!black}
\colorlet{myred}{red!65!black}
\tikzstyle{mydashed}=[xcol,dashed,line width=0.25,dash pattern=on 2.2pt off 2.2pt]
\tikzstyle{axis}=[->,thick] %line width=0.6
\tikzstyle{ell}=[{Latex[length=3.3,width=2.2]}-{Latex[length=3.3,width=2.2]},line width=0.3]
\tikzstyle{dx}=[-{Latex[length=3.3,width=2.2]},darkblue,line width=0.3]
\tikzstyle{ground}=[preaction={fill,top color=black!10,bottom color=black!5,shading angle=20},
                    fill,pattern=north east lines,draw=none,minimum width=0.3,minimum height=0.6]
\tikzstyle{mass}=[line width=0.6,red!30!black,fill=red!40!black!10,rounded corners=1,
                  top color=red!40!black!20,bottom color=red!40!black!10,shading angle=20]
\tikzstyle{spring}=[line width=0.8,blue!7!black!80,snake=coil,segment amplitude=5,segment length=5,line cap=round]
\tikzset{>=latex} % for LaTeX arrow head
\tikzstyle{force}=[->,myred,very thick,line cap=round]
\def\tick#1#2{\draw[thick] (#1)++(#2:0.1) --++ (#2-180:0.2)}

\usepackage{textcomp}

\usepackage{comment}
\newcommand{\fref}[1]{Рис.~\ref{#1}}
\newcommand{\tref}[1]{Таб.~\ref{#1}}
\newcommand{\eref}[1]{(\ref{#1})}



\renewcommand{\labelenumii}{\arabic{enumii}.}

\usepackage{listings}

\usepackage{tikz}
\usetikzlibrary{patterns}



\setcounter{secnumdepth}{6}
\setcounter{tocdepth}{4}

\title{Экзамен по физике}
\author{Составитель: Андрей Рабусов (\url{a.rabusov@tum.de})}
\date{30 мая 2022 г.}

\begin{document}
\maketitle
\par{\bf Задача 1}\par
Два бруска без трения расположены как на рисунке. Внешний брусок с массой
$m_2$ толкают в
сторону внутреннего бруска, масса которого $m_1$.
\begin{figure}[h!]
    \begin{centering}
\begin{tikzpicture}
  \def\H{1}    % wall height
  \def\T{0.3}  % wall thickness
  \def\W{6}  % ground length
  \def\D{0.25} % ground depth
  \def\h{0.6}  % mass height
  \def\w{0.7}  % mass width
  \def\x{1.6}  % mass x position
  \def\y{4.2}  % mass y position
  \def\dx{0.8} % extension
  \def\F{0.8}  % force
  \draw[ground] (0,0) |-++ (-\T,\H) |-++ (\T+\W,-\H-\D) -- (\W,0) -- cycle;
  \draw (0,\H) -- (0,0) -- (\W,0);
  \draw[mass] (\x,0) rectangle++ (\w,\h) node[midway] {$m_1$};
  \draw[mass] (\y,0) rectangle++ (\w,\h) node[midway] {$m_2$};
  \draw[force] (\y,0.5*\h) --++ (-\F,0)
    node[midway,right=1,above=-1] {$\vb{v}$};
\end{tikzpicture}
    \end{centering}
\end{figure}
Сколько всего будет соударений между брусками, а так
же между внутренним бруском и стеной, если
\begin{enumerate}
    \item $m_1 = m_2$?
    \item $m_1 = 1/10 m_2$?
\end{enumerate}

\par{\bf Задача 2}\par
Скучающий космонавт играет на компьютере с тактовой частотой процессора 1
ГГц.  Чему равна частота этого процессора с точки зрения наблюдателя с
Земли, если корабль движется со скоростью $4/5$ скорости света относительно
Земли?
\par{\bf Задача 3}\par
В Стандартной модели физики частиц может ли
\begin{enumerate}
    \item при распаде $\text{b}$-кварка возникнуть $\text{t}$-кварк?
    \item при распаде нейтрона возникнуть мюонное нейтрино?
\end{enumerate}
\par{\bf Задача 4}\par
Найдите значение выражения
\begin{equation*}
\sin 11^\circ + \sin 83^\circ + \sin 155^\circ + \sin 227^\circ +
    \sin 299^\circ
\end{equation*}

\begin{centering}
    {\it
    Задача не считается решённой, если приведён только ответ
    }
\end{centering}

\end{document}
