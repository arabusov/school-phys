\maketitle
\par{\bf Задача 1}\par
Два бруска без трения расположены как на рисунке. Внешний брусок с массой
$m_2$ толкают в
сторону внутреннего бруска, масса которого $m_1$.
\begin{figure}[h!]
    \begin{centering}
\begin{tikzpicture}
  \def\H{1}    % wall height
  \def\T{0.3}  % wall thickness
  \def\W{6}  % ground length
  \def\D{0.25} % ground depth
  \def\h{0.6}  % mass height
  \def\w{0.7}  % mass width
  \def\x{1.6}  % mass x position
  \def\y{4.2}  % mass y position
  \def\dx{0.8} % extension
  \def\F{0.8}  % force
  \draw[ground] (0,0) |-++ (-\T,\H) |-++ (\T+\W,-\H-\D) -- (\W,0) -- cycle;
  \draw (0,\H) -- (0,0) -- (\W,0);
  \draw[mass] (\x,0) rectangle++ (\w,\h) node[midway] {$m_1$};
  \draw[mass] (\y,0) rectangle++ (\w,\h) node[midway] {$m_2$};
  \draw[force] (\y,0.5*\h) --++ (-\F,0)
    node[midway,right=1,above=-1] {$\vb{v}$};
\end{tikzpicture}
    \end{centering}
\end{figure}
Сколько всего будет соударений между брусками, а так
же между внутренним бруском и стеной, если
\begin{enumerate}
    \item $m_1 = m_2$?
    \item $m_1 = 1/10 m_2$?
\end{enumerate}

\par{\bf Задача 2}\par
Скучающий космонавт играет на компьютере с тактовой частотой процессора 1
ГГц.  Чему равна частота этого процессора с точки зрения наблюдателя с
Земли, если корабль движется со скоростью $4/5$ скорости света относительно
Земли?
\par{\bf Задача 3}\par
В Стандартной модели физики частиц может ли
\begin{enumerate}
    \item при распаде $\text{b}$-кварка возникнуть $\text{t}$-кварк?
    \item при распаде нейтрона возникнуть мюонное нейтрино?
\end{enumerate}
\par{\bf Задача 4}\par
Найдите значение выражения
\begin{equation*}
\sin 11^\circ + \sin 83^\circ + \sin 155^\circ + \sin 227^\circ +
    \sin 299^\circ
\end{equation*}

\begin{centering}
    {\it
    Задача не считается решённой, если приведён только ответ
    }
\end{centering}
